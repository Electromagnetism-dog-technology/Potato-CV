
\sectionTitle{项目/实习经历}{\faBarChart}
 
\begin{experiences}

\emptySeparator
\experience
{2019年}{11月} {土豆花样切削大赛冠军 \& 国际土豆切削挑战赛, IPCC}{独自开发}


\\
\faPlayCircleO \textsc{2018.10} & \textsc{中国科学院土豆高级研究所 \& 土豆切削实验室}, \textsc{访问交流} \\ 
\faStopCircleO \textsc{2018.12}  & 
\begin{minipage}[t]{\rightcolumnlength}
	\begin{itemize}
		\item \textbf{工作:} 基于Graph Neural Network 对土豆倾销网络进行建模。
		\item \textbf{成果:}在Deep Graph Library库下复现了Potato-GCN网络,并且在Github上进行了开源,为以后研发AP-GCN网络奠定了基础。
	\end{itemize}
\end{minipage} 

\\
\faPlayCircleO \textsc{2020.04} & \textsc{深圳土豆先进研究院}, \textsc{土豆切削工程师} \\ 
\faStopCircleO \textsc{2020.07}  & 
\begin{minipage}[t]{\rightcolumnlength}
	\begin{itemize}
		\item \textbf{工作:}在实习期间,主要负责研究土豆的各类花样切削方法,并且自动化实现。
		\item \textbf{成果:} 在PyPotato上复现了Potato-Good算法,并且进行了改进,在减少了成本和运算复杂度的同时增加了切削土豆的稳定性。
	\end{itemize}
\end{minipage} \\

\\
\faPlayCircleO \textsc{2020.07} & \textsc{大土豆集团 \& 土豆中台事业群}, \textsc{实习土豆形状研究员} \\ 
\faStopCircleO \textsc{2020.09}  & 
\begin{minipage}[t]{\rightcolumnlength}
	\begin{itemize}
		\item \textbf{工作:}在实习期间,主要负责组内土豆姿态估计相关问题的研究,并且尝试将其在土豆烹饪中更好地结合。
	\end{itemize}
\end{minipage} \\

\end{experiences}

